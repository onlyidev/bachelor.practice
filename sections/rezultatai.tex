\section{Rezultatai, išvados ir pasiūlymai}

\subsection{Darbo rezultatai ir išvados}

\subsubsection*{Pagrindiniai rezultatai}

\begin{itemize}
    \item Išanalizuotos organizacijoje naudojamos objektų saugojimo technologijos ir pasirinkta tinkamiausia specifiniam panaudos atvejui (specifinių programos duomenų archyvavimui) -- \textit{S3 on-prem}.
    \item Programų sistemai suprogramuotas integracijos su \textit{S3 on-prem} sistema komponentas.
    \item Surinkti bei dokumentuoti archyvavimo proceso reikalavimai.
    \item Programų sistemai suprogramuotas archyvavimo proceso komponentas.
    \item Paruoštas sprendimas įdiegtas visose aplinkose.
\end{itemize}

\subsubsection*{Išvados}

\begin{itemize}
    \item Architektūrinis sprendimas integraciją su \textit{S3 on-prem} įgyvendinti kaip atskirą komponentą yra lankstus tiek panaudos atvejų, tiek duomenų, tiek infrastruktūros požiūriu. Pavyzdžiui, kuriant bei įgyvendinant naujus procesus, generuojančius archyvavimui tinkamus duomenis, galima naudotis jau esamu komponentu ir archyvavimą įtraukti į pradinį proceso įgyvendinimą.
    \item Archyvavimo procesas sprendžia gan paprastą problemą ir iš esmės nėra sudėtingas, tačiau surinkus reikalavimus iš naudotojų, sprendimų architektų, organizacijos vidinių tvarkų bei teisės aktų tampa akivaizdu, jog proceso įgyvendinimas -- netrivialus.
    \item Atlikus sprendimo diegimą visose aplinkose galima teigti, jog pagrindinis praktikos tikslas -- \enquote{sukurti automatizuotą versijavimo įrašų archyvavimo sprendimą} -- įgyvendintas.
    \item Atlikus skaičiavimus su duomenų bazėje esančiu duomenų kiekiu nustatyta, jog po \sim  2 mėn. duomenų bazės versijavimo įrašų lentelė užims \sim 20 kartų mažiau vietos. Tai leidžia teigti, jog pagrindinis praktikos uždavinys -- \enquote{atlaisvinti vietą duomenų bazėje ir sumažinti jos apkrovą} -- pasiektas.
\end{itemize}

\subsection{Praktikos darbo privalumai ir trūkumai}

\subsubsection*{Privalumai}
\begin{itemize}
    \item Praktikos užduotis apėmė visas programų sistemų inžinerijos proceso dalis -- reikalavimų rinkimą, analizę, sprendimo architektūros kūrimą, programavimą, testavimą, diegimą ir dokumentavimą.
    \item Praktikos užduotis buvo atliekama \textit{Agile} komandoje dalyvaujant visose komandos veiklose ir procesuose. Tai atitiko realias darbo sąlygas.
\end{itemize}

\subsubsection*{Trūkumai}

Profesinė praktika atitiko jai keliamus reikalavimus ir lūkesčius, trūkumų nenustatyta.

\subsection{Žinių įvertinimas}
Profesinės praktikos metu įgytos ar pagilintos šios žinios ir įgūdžiai:
\begin{itemize}
    \item Objektų saugojimo \angl{object storage} technologijos, jų panaudos atvejai \angl{use cases}.
    \item S3 sąsaja \angl{S3 API}.
    \item \textit{Spring Boot} karkasas \angl{framework}.
    \item \textit{Web MVC} karkasas.
    \item \textit{Agile} metodologija.
    \item Komandinis darbas.
    \item Automatinis testavimas.
    \item Pakeitimų valdymas.
    \item Programų sistemų konfigūracijos valdymas ir CaC \angl{Configuration as Code}.
\end{itemize}

Praktikos metu, lyginant su universitete gautomis žiniomis, įgyta daugiau specifinių ir konkretesnių žinių, tiesiogiai taikomų praktikoje. Pavyzdžiui, objektų saugimo technologijos nėra dėstomos universitete, tačiau tai tėra specifinio duomenų saugojimo sprendimo abstrakcija, o pagrindinės duomenų saugojimo technologijos -- duomenų bazės -- yra dėstomos universitete.

Panaši situacija yra ir su karkasais -- tai yra specifinės technologijos, apie kurias universitete susipažįstama objektinio programavimo ir programų sistemų kūrimo kursuose.

Apie \textit{Agile} metodologiją, komandinį darbą bei automatinį testavimą yra universiteto paskaitose kalbama plačiai ir išreikštinai, tad praktikoje šios žinios buvo pritaikytos ir įtvirtintos. 

Pakeitimų valdymas ir programų sistemų konfigūracijos valdymas yra itin svarbūs procesai, kuriuose dalyvauja kiekvienas komandos narys, tačiau universitete apie šiuos procesus buvo tik užsiminta. Praktikos metu įsigilinti į šiuos procesus nėra pakankamai laiko, tad įgytos tik pradinės žinios.

\subsection{Rekomendacijos universitetui}

\subsubsection*{DevOps / DevSecOps}

Dažnai už programinės įrangos kūrimą atsakingos komandos taip pat yra dalinai ar visiškai atsakingos ir už programinės įrangos infrastruktūros valdymą. Programų sistemų kūrėjams neretai tenka valdyti programinės įrangos konfigūraciją, tinklo, duomenų bazių prieigą, atlikti telemetriją bei analizuoti programinės įrangos veikimą. Tai yra operacinė \textit{DevOps} dalis. Be to, šių dienų praktikos rodo, jog saugumo užtikrinimas taip pat neretai pereina iš vienos centrinės saugumo komandos organizacijoje į kūrimo ir palaikymo komandas -- \textit{DevSecOps}. Taigi, studentai turėtų būti ne tik supažindinami su šiomis praktikomis, bet ir ruošiami kaip programų sistemų kūrimo, palaikymo, infrastruktūros, operacijų bei saugumo specialistai. 

\subsubsection*{Debesijos technologijos}

Debesijos \angl{cloud} technologijos per pastarąjį dešimtmetį tapo itin populiarios ir dažnai įrašomos IT įmonių strategijose. Debesija ne tik suteikia daug galimybių, bet ir kelia tam tikrus reikalavimus programų sistemų architektūrai, į kuriuos būtina atsižvelgti kuriant praktiškai bet kokią verslui vertę nešančią sistemą. Autoriaus nuomone, universitetas turėtų supažindinti studentus su debesijos technologijomis detaliai aptariant ir pagrindinius privalomus priimti sprendimus kuriant programų sistemas. Taip pat universitetas turėtų suteikti žinias ir galimybę studentams išbandyti debesijos technologijas praktiškai.