\section{Įmonės apibūdinimas}

\subsection{Įmonės organizacinė struktūra}
\SEB~yra vienas didžiausių Lietuvoje veikiančių komercinių bankų, priklauso Švedijos bankininkystės grupei \textit{Skandinaviska Enskilda Banken} (SEB) ir yra priskiriamas \textit{Baltijos} divizijai. 

\PDT~veikia \textit{Baltijos} divizijoje ir sujungia visų trijų Baltijos šalių \textit{SEB} darbuotojus į tarptautines komandas, atsakingas už IT infrastruktūros ir produktų palaikymą ir plėtrą.

\PD~-- \textit{Produktų plėtros ir technologijų centro} departamentas, atsakingas už produktus ir procesus. Čia dirba IT architektai, programų sistemų kūrėjai, testuotojai, verslo analitikai, projektų vadovai ir kiti specialistai. Šio departamento komandos dirba sekdamos \textit{Agile} \cite{cohenIntroductionAgileMethods2004} metodologiją, tad darbo procesai yra dinamiški ir greitai keičiasi reaguojant į pokyčius.

\subsection{Produktų plėtros departamento teikiamos paslaugos}

\PD~nuolat bendradarbiauja su kitais banko departamentais ir klientais, siekdami sukurti ir palaikyti aukštos kokybės produktus. Pagrindinės šio departamento veiklos sritys yra:
\begin{itemize}
    \item \textbf{Produktų plėtra}
    \begin{itemize}
        \item Naujų produktų kūrimas banko klientams ir darbuotojams.
        \item Esamų produktų tobulinimas.
        \item Programų sistemų palaikymas.
    \end{itemize}
    \item \textbf{Procesų optimizavimas} -- esamų procesų tobulinimas ir automatizavimas.
\end{itemize}

\subsection{Sudarytos darbo sąlygos}
Praktikos vietoje galioja atviro ofiso \angl{open office} principas -- nėra atskirų kabinetų, darbuotojai dirba vienoje erdvėje. Tai skatina bendradarbiavimą ir komunikaciją. Kiekvienoje atviroje darbo vietoje yra ergonomiška kėdė, pakeliamas stalas bei du monitoriai.